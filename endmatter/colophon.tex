%!TEX root = ../dissertation.tex
\newpage

% If you do want an image in the colophon:
% \begin{figure}
%     \vspace{20pt}
%     \centering
%     \hspace*{-32pt}
%     \includegraphics[width=0.42\textwidth]{endmatter/colophon.png}
% \end{figure}

% If you don't want an image in the colophon:
% \vspace*{200pt}

\begin{center}
    \parbox{400pt}{\lettrine[lines=3,slope=-2pt,nindent=-4pt]{\textcolor{SchoolColor}{T}}{his thesis was typeset} using \LaTeX, originally developed by Leslie Lamport and based on Donald Knuth's \TeX. The body text is set in 11 point Egenolff-Berner Garamond, a revival of Claude Garamont's humanist typeface. A template that can be used to format a PhD dissertation with this look \textit{\&} feel has been released under the permissive \textsc{agpl} license, and can be found online at \href{https://github.com/suchow/Dissertate}{github.com/suchow/Dissertate} or from its lead author, Jordan Suchow, at \href{mailto:suchow@post.harvard.edu}{suchow@post.harvard.edu}. The template used in this dissertation is created based on Harvard Template, which was previously used in the \href{https://eprints.lib.hokudai.ac.jp/dspace/bitstream/2115/70225/1/Zhai_HongJie.pdf}{dissertation of Hongjie Zhai}, who is a former student of Hokkaido University. The logo of Hokkaido University is used under the \href{https://www.hokudai.ac.jp/introduction/information/symbol/}{University Logo Guideline}.}
\end{center}
